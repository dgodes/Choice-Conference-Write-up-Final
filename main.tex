\documentclass{article}
\usepackage{lmodern}
\usepackage[top=1in, bottom=1in, left=1in, right=1in]{geometry}
\usepackage{natbib}
\usepackage{amsmath}
\usepackage{authblk}
\usepackage{setspace}
\usepackage{graphicx}
\usepackage{hyphenat}
\usepackage{breqn}
\usepackage{caption}
\usepackage{dcolumn}
\usepackage{parskip}
\usepackage{adjustbox}
\usepackage{amsthm}
%\usepackage{amsfonts}
\usepackage{booktabs}
\usepackage{siunitx}
\usepackage{amssymb}
\usepackage{booktabs}
\usepackage{placeins}
\usepackage{float}
\usepackage{bbm}
\newcommand*\rfrac[2]{{}^{#1}\!/_{#2}}%running fraction with slash
\newtheorem{prop}{Proposition}
\newtheorem{corollary}{Corollary}
\newtheorem{theorem}{Theorem}
\newtheorem{assumption}{Assumption}
\newtheorem{definition}{Definition} 
\newtheorem{lemma}{Lemma}

\date{}
\title{Politics, Persuasion, and Truth}
\author{David Godes, Dina Mayzlin, Richard Kotchmar, Pinar Yildirim, Judith Chevalier, Neeru Paharia, Lingling Zhang, Odilon Camara, Chris Hydock, Sarah Moshary, Seon Hye (Claire) Lim, Doug Chung, Nils Wernerfelt, and Ali Yurukoglu\footnote{David Godes and Dina Mayzlin were the organizers, all else are listed in alphabeticla order.}}

\begin{document}
\maketitle
 \begin{abstract}
ADD IT HERE 
\end{abstract}
{\it Keywords:} politics, persuasion


%\setcounter{page}{1}

\begin{spacing}{2}
 \setlength{\parindent}{0.5cm}
% \fontsize{11}{11} \selectfont
\section{Introduction}
Test
As the study of government, politics, like marketing, has long had an ambivalent relationship with the concept of Truth. Thomas Jefferson famously said, ``It is error alone which needs the support of government. Truth can stand by itself.'' The notion that Truth requires no ``support'' - that it is, in essence, self-evident - is appealing in theory and is wholly consistent with the earliest conceptualizations of democracy. However, there remains a range of theoretical, empirical and practical obstacles in the way of Jefferson's claim that Truth is inevitable. In this paper, we report on, summarize, and extend our discussions at the Choice Symposium. While our session was diverse and far-reaching, we attempt to organize our ideas around several dimensions of Truth at it relates to politics with the objective of spurring additional research on these topics:

What do we mean by ``Truth,'' within the context of politics and, more broadly, political processes?

To what extent is Truth a desirable or highly-valued outcome? Does more truth lead to higher welfare?

What mechanisms facilitate - or obstruct - the production of truthful outcomes?


\section{What is truth?}\label{Sec: truth_def}
The concept of truth is most, perhaps only, relevant in a context of uncertainty. Imagine, for example, the currently highly-debated issue of immigration in the United States. One particularly-thorny (though far from the only) question in this debate is the optimal level of immigration. Let this optimum be given by $\theta$. Presumably, such an optimum exists but, in this case, it is effectively impossible to state with any precision. Thus, each individual forms her own belief about $\theta$ which, in the aggregate, is often represented as a distribution on the domain of $\theta: G(\theta)$.  Why would there be heterogeneity in each individuals' assessments of $\theta$? While there are many sources, two critical drivers are (i) the information available to each individual $i$, and (ii) individual $i$'s preferences. The impact of information on beliefs is both well studied and somewhat intuitive. However, it is worth acknowledging that the spread of the aggregate distribution is likely to be a function of the diversity of reporting in the media. 

In neoclassical economic terms, beliefs and preferences are distinct constructs.  Preferences reflect an ordering over goods while beliefs reflect (subjective) likelihoods regarding possible states of the world. However, less attention has been paid to the manner in which these preference orderings are established. While traditional economic theory usually specifies preferences as being ``exogenously endowed,'' even Stigler and Becker (1977) (''De gustibus non est disputandum'') resort to states of the world, and thus beliefs, as an explanation for actual behavior. More precisely, to the extent that one's exogeneously-endowed preferences drive media consumption and social network formation, then they will drive as well information availability and thus, at least indirectly, beliefs. Moreover, the psychological research on motivated reasoning suggests a more-direct link between preferences and beliefs.

One may also consider a meta-level perspective on truth: the 


This is important for a number of reasons but most importantly as it relates to the concept of the ``Wisdom of Crowds'' (Surowiecki 2004). In his book, Surowiecki takes an axiomatic approach to making the argument that, in essence, one can identify Truth by aggregating beliefs. Returning to the above points, then, a policy maker would do well to poll the public on their beliefs regarding immigration to learn the Truth. Unfortunately, as we will argue below, many of his axioms are not satisfied in practice.

a. Polarization
\item Is it happening?
\item Why?
Doesn't a WoC model imply a uni-model distribution?
\item Pinar
\item Ali: it's media fault

b. Democracy and Voting as representing will of the people
\item What mechanisms would actually represent the actual will?
\item Referenda - Why different from representative democracy?
\item What (if anything) does this say about such methods in marketing?

C. Bayesian Persuasion (? Not sure where this goes)

\section{Is truth itself a good?}\label{Sec: truth_good}
In the impossibly-expansive neoclassical economic definition, a ``good'' is defined by the utility it delivers. One very-simple answer to this question is that uncertainty leads to inefficient outcomes. See, for example the Myerson-Satterthwaite theory on two-sided exchange. In such a context, the reduction in uncertainty would lead to higher welfare. However, does this hold more generally? If we see a highly-dispersed  as evidence of less Truth, does it follow that welfare would improve by narrowing this distribution?

a. Polarization again (Pinar) - Defining this as a multi-modal distribution of beliefs about the Truth, it would seem that perspective that polarization is ``bad'' is tantamount to saying that variance on beliefs about Truth is welfare decreasing. Is this true? Again, information theory would argue these diverse views are good.

b. Diverse opinions (Nils): It is interesting to compare this to the rise in discussion around tolerance, diversity and inclusion (relates to Nils' work). These ideas seem to emanate from the opposite idea - that differences in beliefs about Truth are good and positive and are facilitated by social media.


C. Bayesian Persuasion (? Not sure where this goes)

\section{What mechanisms facilitate the truth?}\label{Sec: mechanisms}

\item (Dave) - Competition, fact checking
\item Platforms/Social Media - Curation/Algorithms (Dina)
\item Corporate participation (Chris Neeru)
\item Competition in advertising marketplace (Lingling, Doug)


\section{Conclusion} \label{Sec: conclusion}
Blah blah   



\pagebreak


\pagebreak
\bibliographystyle{apa}
\bibliography{Blogger_lib}
\end{spacing}





\end{document}